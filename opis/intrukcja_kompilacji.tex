\documentclass[a4paper]{article}
% Kodowanie latain 2
%\usepackage[latin2]{inputenc}
\usepackage[T1]{fontenc}
% Można też użyć UTF-8
\usepackage[utf8]{inputenc}

\usepackage{enumerate}
% Język
\usepackage[polish]{babel}
% \usepackage[english]{babel}
\usepackage{verbatim}
% Rózne przydatne paczki:
% - znaczki matematyczne
\usepackage{amsmath, amsfonts}
% - wcięcie na początku pierwszego akapitu
\usepackage{indentfirst}
% - komenda \url 
\usepackage{hyperref}
% - dołączanie obrazków
\usepackage{graphics}
% - szersza strona
\usepackage[nofoot,hdivide={2cm,*,2cm},vdivide={2cm,*,2cm}]{geometry}
\frenchspacing
% - brak numerów stron
\pagestyle{empty}

% dane autora
\author{Karol Kuczmarz, 308845}
\title{Kropki - instrukcja kompilacji}
\date{\today}

% początek dokumentu
\begin{document}
\maketitle
\section{Jak skompilować program?}
Program składa się z wielu modułów, dlatego został napisany plik "Makefile", z którego skorzysta polecenie make, które wpiszemy w terminal, gdy będziemy znajdowali się w folderze z plikami źródłowymi i plikiem "Makefile".
\section{Niestandardowe biblioteki, z których korzyszta program.}
Program korzysta z infterfejsu okienkowego GTK+ oraz z biblioteki Cairo do grafiki wektorowej. Ponadto wykorzystuje pliki kolejkowe fifo do komunikacji między dwiema działającymi kopiami programu.
Lista niestandardowych bibliotek:
\begin{enumerate}
\item gtk/gtk.h
\item cairo.h
\item sys/stat.h
\item unistd.h
\item fcntl.h
\end{enumerate}
\end{document}
