\documentclass[a4paper]{article}
% Kodowanie latain 2
%\usepackage[latin2]{inputenc}
\usepackage[T1]{fontenc}
% Można też użyć UTF-8
\usepackage[utf8]{inputenc}

\usepackage{enumerate}
% Język
\usepackage[polish]{babel}
% \usepackage[english]{babel}
\usepackage{verbatim}
% Rózne przydatne paczki:
% - znaczki matematyczne
\usepackage{amsmath, amsfonts}
% - wcięcie na początku pierwszego akapitu
\usepackage{indentfirst}
% - komenda \url 
\usepackage{hyperref}
% - dołączanie obrazków
\usepackage{graphics}
% - szersza strona
\usepackage[nofoot,hdivide={2cm,*,2cm},vdivide={2cm,*,2cm}]{geometry}
\frenchspacing
% - brak numerów stron
\pagestyle{empty}

% dane autora
\author{Karol Kuczmarz, 308845}
\title{Gra w kropki - opis projektu}
\date{\today}

% początek dokumentu
\begin{document}
\maketitle
\section{Wstęp i krótka charkterystyka}
Gra będzie przeznaczona dla dwóch użytkowników. Będzie należało uruchomić dwie kopie programu, każdą z odpowiednim argumentem uruchomienia programu.  Dwa uruchomione programy będą się ze sobą komunikowały poprzez plik kolejkowy.
\section{Interakcja użytkownika z programem}
\begin{enumerate}
\item W czasie rozgrywki użytkownicy naprzemiennie wybierają kropkę, którą chcą zająć. Użytkownik będzie czekał na ruch przeciwnika.
\item Gracz może opuścić rozgrywkę w dowolnym momencie - poddać się lub wspólnie z przeciwnikiem zadecydować o podliczeniu punktów.
\end{enumerate}
\section{Realizowane przez program funkcje}
\begin{enumerate}
\item Zbieranie koorynatów kliknięcia myszki i ustalanie, które pole zostało wybrane.
\item Sprawdzanie, czy ruch użytkownika jest zgodny z zasadami gry.
\item Sprawdzanie, czy ruch użytkownika tworzy bazę (wielokąt) i tworzenie tej bazy
\begin{enumerate}
\item Ustalanie wyglądu bazy.
\item Sprawdzenie, które punkty są w środku.
\item Policzenie i przyznanie punktów odpowiedniemu graczowi.
\item Ustawienie niezajętych kropek wewnątrz jako niemożliwych do zajęcia. 
\end{enumerate}
\item Umieszczanie informacji o ruchu użytkownika w pliku kolejkowym.
\item Rysowanie planszy i jej aktualizacja, tzn. nanoszenie kropek i łączenie ich liniami w przypadku istnienia bazy.
\item Umożliwiane użytkownikowi poddania gry lub skierowania propozycji zakończenia rozgrywki. 
\end{enumerate}

\end{document}
